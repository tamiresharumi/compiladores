\section{Visão geral}
O compilador desenvolvido aceita uma versão simplificada do Pascal.
Conceitualmente, o compilador possui quatro etapas:

\begin{itemize}
	\item Análise léxica;
	\item Análise sintática;
	\item Análise semântica;
	\item Geração de código.
\end{itemize}

Apesar de serem quatro etapas que podem estar logicamente distintas, no
compilador desenvolvido algumas delas estão sobrepostas.

A análise léxica é realizada por um programa gerado pelo comando \texttt{flex},
a partir do arquivo fonte \texttt{main.lex}. O código gerado pelo \texttt{flex}
é responsável por fornecer \emph{tokens} ao anaĺisador léxico e, quando
adequado, o valor de alguns destes \emph{tokens}. Por exemplo, quando o
analisador léxico encontra um número inteiro, com o formato \texttt{DIGITO+},
retorna além do \emph{token} \texttt{TOKEN\_LITERAL\_INTEIRO} o valor
convertido para um inteiro que é armazenado na variável \texttt{yylval}. Essa
variável é criada pelo \texttt{bison}, que é responsável pela análise
sintática.

A ponte entre o analisador sintático e o analisador léxico é a função
\texttt{yylex}, criada pelo \texttt{flex}, que retorna o token da entrada e
realiza a conversão do valor de entrada quando necessário, por exemplo quando
encontra um número inteiro ou um número real.

A análise sintática é realizada totalmente pelo \texttt{bison}, que gera um
analisador sintático a partir da gramática descrita no arquivo \texttt{sint.y}.
Durante a análise sintática, não é construída nenhuma árvore de derivação ou
outra estrutura que represente o programa, todos as etapas seguintes são
executadas juntamente à analise sintática.

A verificação de erros sintáticos é realizada com o auxílio da produção
especial \texttt{error}, fornecida pelo próprio \texttt{bison} para realizar o
tratamento de erros. Quando um erro é detectado, uma mensagem é emitida na
saída padrão e em seguida tenta-se deixar o analisador sintático em um estado
que o permita continuar a processar o arquivo e encontrar outros erros, caso
existam.

Durante a análise sintática, também é realizada a análise semântica. Nesta
última, são verificados, entre outros:
\begin{itemize}
	\item o uso de variáveis não declaradas;
	\item a tentativa de atribuição de valores para constantes;
	\item chamadas a procedimentos não existentes;
	\item a concordância de quantidade, número e ordem de parâmetros de
	procedimentos;
	\item a compatibilidade de tipos entre variáveis.
\end{itemize}

Este último item pode ser observado no trecho de código a seguir, que apresenta
as ações semânticas associadas à regra \texttt{expressao}, nas quais são
mostradas que o tipo de uma expressão depende do tipo de seus elementos.

\begin{verbatim}
expressao:
        termo
        {
            $$ = $1;
        }
    |   expressao op_termo termo 
        {
            if ($1.tipo != $3.tipo)
                $$ = simbolo_variavel(TIPO_FLOAT);
            else
                $$ = simbolo_variavel($1.tipo);
            if (SHOULD_I_GENERATE_CODE)
            {
                if ($2 == TOKEN_SOMA)
                    C.push_back("SOMA");
                else
                    C.push_back("SUBT");
            }
        }
    ;
\end{verbatim}

A geração de código é feita também dentro das ações semânticas da gramática,
como pode ser observado no exemplo acima. A variável global \texttt{C}
representa o vetor de código no qual estão todas as instruções do programa
final.

Toda a compilação é realizada em apenas uma etapa, sem ser necessário ler o
código ou o vetor de programa mais de uma vez. Mais detalhes são apresentados
na seção~\ref{sec:decisoes}.

