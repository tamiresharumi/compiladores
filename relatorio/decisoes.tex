\section{Decisões de Projeto e Implementação}
\label{sec:decisoes}

Para a última parte do trabalho, não é permitido chamadas de
procedimento dentro de um procedimento. Consequentemente, procedimentos
recursivos também não são permitidos. Caso isso aconteça, um erro semântico é
disparado. Essa decisão foi tomada tendo como ponto chave o conjunto de
instruções dado, que dificulta o endereçamento das variáveis nos procedimentos.

Outra decisão tomada foi a de não emitir mensagem de erro quando há variáveis
de tipos diferentes dentro de um \texttt{readln} ou um \texttt{writeln}, já que
a instrução para ler ou imprimir variáveis é a mesma, independente do tipo.

Há uma tabela de símbolos para cada escopo, ou seja, uma tabela para o programa
e uma tabela para cada procedimento declarado. Cada tabela de procedimento
possui um ponteiro para a tabela, para acesso das variáveis alocadas um escopo
acima.

Para as operações em que são permitidos operandos de tipos diferentes 
(\texttt{INT} e \texttt{FLOAT}), o resultado da operação será um \texttt{FLOAT}.

A geração de código é feita juntamente com a análise semântica, em apenas uma
etapa. Para solucionar o problema de endereçamento dos desvios, foi criado uma
pilha como ferramenta auxiliar, para armazenar os endereços de retorno ou
endereços de atualização. Tal estrutura foi escolhida pela característica de
regras da gramática: como uma regra menor deve ser sintetizada antes de uma
regra maior que a contenha, o último elemento colocado será o primeiro a ser
retirado, caracterizando assim uma pilha.

Notas sobre a gramática:
\begin{itemize}
	\item Os argumentos de um procedimento são separados por ponto-e-vírgula,
	diferentemente dos exemplos dados em sala, em que eram separados por
	vírgula.
\end{itemize}

Não proc dentro de proc
Não proc recursivo
Proc com args separados por ;
readln pega qqr valor já que o conj inst eh o msm
msm coisa p writeln
tabela de simbolos local p cd proc
antes de else nao pode ;
mult float x int tem res float

