\section{Compilação e Execução}
	Para compilar os arquivos, basta executar o comando \texttt{make}. Para
executar o arquivo, deve-se usar a seguinte linha de comando:
\begin{verbatim}
	./trabalho3 <nome do arquivo de entrada> <nome do arquivo de saída>
\end{verbatim}

\subsection{Exemplos}
A seguir, são apresentados alguns exemplos e o código gerado pelo compilador.
As linhas da saída do compilador estão numeradas para facilitar a localização
dos alvos das instruções de desvio.
\begin{multicols}{2}
\paragraph{proc1.pas}~
\scriptsize
\lstinputlisting[language=pascal]{../exemplos/relatorio/proc1.pas}
\lstinputlisting[numbers=left]{../exemplos/relatorio/proc1-output}
\normalsize

\line(1,0){100}
\paragraph{proc2.pas}~
\scriptsize
\lstinputlisting[language=pascal]{../exemplos/relatorio/proc2.pas}
\lstinputlisting[numbers=left]{../exemplos/relatorio/proc2-output}
\normalsize

\line(1,0){100}
\paragraph{if.pas}~
\scriptsize
\lstinputlisting[language=pascal]{../exemplos/relatorio/if.pas}
\lstinputlisting[numbers=left]{../exemplos/relatorio/if-output}
\normalsize

\line(1,0){100}
\paragraph{while.pas}~
\scriptsize
\lstinputlisting[language=pascal]{../exemplos/relatorio/while.pas}
\lstinputlisting[numbers=left]{../exemplos/relatorio/while-output}
\normalsize

\line(1,0){100}
\paragraph{repeat.pas}~
\scriptsize
\lstinputlisting[language=pascal]{../exemplos/relatorio/repeat.pas}
\lstinputlisting[numbers=left]{../exemplos/relatorio/repeat-output}
\normalsize

\end{multicols}
